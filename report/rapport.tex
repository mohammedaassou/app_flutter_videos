\documentclass[11pt,a4paper]{article}

	itle{Flutter Project Exam\\\large Mini CV \& Video Search App}
\usepackage[utf8]{inputenc}
\usepackage{lmodern}
\usepackage{graphicx}
\usepackage{hyperref}
\usepackage{xcolor}
\usepackage{listings}

\hypersetup{
  colorlinks=true,
  linkcolor=blue,
  urlcolor=blue,
  citecolor=blue
}

\author{Mohamed Aassou\\Supervisor: Prof. Ayad}
\date{\today}

\lstset{
  basicstyle=\ttfamily\small,
  breaklines=true,
  frame=single,
  rulecolor=\color{black!15},
  columns=fullflexible
}

\begin{document}

\maketitle

\begin{abstract}
This document is a short technical report (\textit{rapport}) for a Flutter application that combines a mini CV/portfolio UI with API-driven features: a video search experience and a GitHub repositories list. The goal is to demonstrate UI composition, navigation, simple animations, and REST API integration in Flutter.
\end{abstract}

\tableofcontents

\section{Project Overview}
\subsection{Objective}
The application showcases:
\begin{itemize}
  \item A portfolio-style home screen (mini CV).
  \item A video search feature backed by a public video API.
  \item A GitHub repositories list powered by the GitHub REST API.
  \item A simple profile/settings UI.
\end{itemize}

\subsection{Main Screens}
The app uses a bottom navigation bar with four tabs:
\begin{enumerate}
  \item \textbf{Home}: profile image + skills + projects.
  \item \textbf{Video Search}: query input, results list, and navigation to details.
  \item \textbf{GitHub Repos}: fetch and display public repositories for a user.
  \item \textbf{Profile}: tabbed interface (About / Settings / Links).
\end{enumerate}

\section{Technology Stack}
\begin{itemize}
  \item \textbf{Flutter/Dart}: UI framework and language.
  \item \textbf{HTTP}: REST calls via the \texttt{http} package.
  \item \textbf{Video playback}: \texttt{video\_player}.
  \item \textbf{Animations}: \texttt{animations} (Material motion) + custom staggered transitions.
  \item \textbf{Navigation}: \texttt{curved\_navigation\_bar} for bottom navigation.
\end{itemize}

\section{Architecture and Code Organization}
The codebase is organized by responsibility:
\begin{itemize}
  \item \texttt{lib/models/}: data models (e.g. GitHub repo and video objects).
  \item \texttt{lib/services/}: thin API clients for HTTP calls.
  \item \texttt{lib/*.dart}: screens/pages and UI widgets.
  \item \texttt{assets/}: images and fonts.
\end{itemize}

This project primarily uses \texttt{FutureBuilder} for async UI updates (no additional state management framework).

\section{API Integration}
\subsection{Video Search API}
The current implementation calls Pixabay's video API endpoint:
\begin{quote}
\url{https://pixabay.com/api/videos}
\end{quote}
The API key and base URL are configured in \texttt{lib/env.dart}.

\subsection{GitHub Repositories API}
The GitHub repositories tab requests public repositories via:
\begin{quote}
\url{https://api.github.com/users/mohammedaassou/repos}
\end{quote}
It parses the response into a \texttt{GitHubRepo} model and renders a list with basic repo metadata (stars, forks, language, and last update date when available).

\section{UI and Interaction Details}
\subsection{Transitions and Animations}
The app uses:
\begin{itemize}
  \item Staggered fade/scale for list items (helper widget in \texttt{lib/widgets/staggered\_fade\_scale.dart}).
  \item Fade-through transitions when opening a video item (Material motion \texttt{OpenContainer}).
\end{itemize}

\subsection{Video Details and Playback}
The video details page initializes a network video controller, starts playback, and provides a tap-to-play/pause overlay.

\section{Build and Run}
\subsection{Run the Flutter app}
From the project root:
\begin{lstlisting}
flutter pub get
flutter run
\end{lstlisting}

\subsection{Build this report (PDF)}
From the \texttt{report/} directory:
\begin{lstlisting}
pdflatex rapport.tex
\end{lstlisting}
Run \texttt{pdflatex} twice if you want the table of contents to update.

\section{Assets}
A profile image is included as an app asset:
\begin{quote}
\texttt{assets/mohamed.png}
\end{quote}
If you want to include it in this report (optional), you can uncomment the figure below.

% \begin{figure}[h]
%   \centering
%   \includegraphics[width=0.25\textwidth]{../assets/mohamed.png}
%   \caption{Profile image asset used by the app.}
% \end{figure}

\section{Screenshots}
The following screenshots are stored in the \texttt{demo/} folder and document the main screens of the application.

\clearpage

\begin{figure}[!htbp]
  \centering
  \includegraphics[width=0.32\textwidth]{../demo/1-homepage-minicv.png}
  \hfill
  \includegraphics[width=0.32\textwidth]{../demo/2-videos-tab.png}
  \hfill
  \includegraphics[width=0.32\textwidth]{../demo/3-view-video.png}
  \caption{Home (Mini CV), Video Search, and Video Details.}
\end{figure}

\begin{figure}[!htbp]
  \centering
  \includegraphics[width=0.32\textwidth]{../demo/4-repos.png}
  \hfill
  \includegraphics[width=0.32\textwidth]{../demo/5-profile.png}
  \caption{GitHub Repos and Profile tabs.}
\end{figure}

\section{Limitations and Improvements}
\begin{itemize}
  \item The UI mentions “Pexels” in a few places, while the current API endpoint is Pixabay.
  \item Error handling is functional (retry + message), but could be improved with richer empty/error states.
  \item A dedicated state management solution (Provider/Riverpod/Bloc) could be introduced if the app grows.
\end{itemize}

\section{References}
\begin{itemize}
  \item Flutter documentation: \url{https://docs.flutter.dev/}
  \item GitHub REST API: \url{https://docs.github.com/en/rest}
  \item Pixabay API (Videos): \url{https://pixabay.com/api/docs/}
\end{itemize}

\end{document}
