% Rapport de projet Flutter (en français)
\documentclass[12pt,a4paper]{report}

% Encodage et langue
\usepackage[T1]{fontenc}
\usepackage[utf8]{inputenc}
\usepackage[french]{babel}

% Mise en page et visuels
\usepackage{geometry}
\geometry{margin=2.5cm}
\usepackage{graphicx}
\usepackage{xcolor}
\usepackage{hyperref}
\hypersetup{colorlinks=true, linkcolor=blue, urlcolor=blue, citecolor=blue}
\usepackage{enumitem}

% En-têtes/pieds de page
\usepackage{fancyhdr}
\pagestyle{fancy}
\fancyhf{}
\lhead{Rapport de projet Flutter}
\rhead{Université Mundiapolis}
\cfoot{\thepage}

\begin{document}

% Page de garde
\begin{titlepage}
  \centering
  \vspace*{1cm}
  % Logo Mundiapolis (présent dans le dossier demo)
  \includegraphics[width=0.6\textwidth]{../demo/logo_mundia.png}\\[0.8cm]

  {\Huge \textbf{Mini CV + App Flutter pilotée par API}}\\[0.6cm]
  {\Large \textbf{Rapport de projet}}\\[1.2cm]

  \begin{tabular}{rl}
    \textbf{Université :} & Université Mundiapolis \\
    \textbf{Encadrant :} & Prof. Ayad \\
    \textbf{Étudiant :} & \textit{Mohamed AASSOU} \\
    \textbf{Filière :} & \textit{2ANCI - Génie Informatiquee} \\
    \textbf{Année :} & 2026 \\
  \end{tabular}

  \vfill
  \large Casablanca, Maroc\\[0.3cm]
  \today
\end{titlepage}

\tableofcontents
\newpage

\chapter{Introduction}
Ce rapport présente une application Flutter démonstrative combinant un \textit{mini CV} et des fonctionnalités connectées à des APIs publiques. L'application offre une navigation par onglets (\textit{bottom navigation}) pour explorer :
\begin{itemize}[leftmargin=1.2cm]
  \item \textbf{Accueil} : page de portfolio avec animations légères.
  \item \textbf{Recherche Vidéo} : recherche et lecture de vidéos depuis l'API \textit{Pixabay Video}.
  \item \textbf{Dépôts GitHub} : liste des dépôts publics de l'utilisateur \texttt{mohammedaassou}.
  \item \textbf{Profil} : page de profil à onglets (\textit{About / Settings / Links}).
\end{itemize}

Les objectifs pédagogiques incluent la manipulation de widgets Flutter, l'appel à des services REST, la structuration d'un projet, et l'intégration de médias (vidéo) et animations.

\chapter{Aperçu de l'application}
L'entrée de l'application se situe dans \texttt{lib/main.dart}. La navigation principale (\textit{CurvedNavigationBar}) est gérée dans \texttt{lib/principale\_page.dart}. Les différentes vues sont placées dans \texttt{lib/pages/} et les modèles/clients HTTP dans \texttt{lib/models/} et \texttt{lib/services/}.

\section{Onglets et fonctionnalités}
\begin{itemize}[leftmargin=1.2cm]
  \item \textbf{Accueil} (\texttt{home\_page.dart}) : mini CV (compétences, projets) avec transitions \texttt{animations}.
  \item \textbf{Recherche Vidéo} (\texttt{video\_search\_page.dart}) : champ de recherche, liste des résultats, navigation vers \texttt{video\_detail\_page.dart} pour la lecture via \texttt{video\_player}.
  \item \textbf{Dépôts GitHub} (\texttt{repos\_page.dart}) : récupération des dépôts publics via \texttt{github\_api.dart} et affichage (nom, description, langue, etc.).
  \item \textbf{Profil} (\texttt{profile\_page.dart}) : onglets \textit{About}, \textit{Settings}, \textit{Links}.
\end{itemize}

\chapter{Architecture et technologies}
\section{Organisation du projet}
\begin{itemize}[leftmargin=1.2cm]
  \item \texttt{lib/main.dart} : point d'entrée; configuration de thème et routes.
  \item \texttt{lib/principale\_page.dart} : barre de navigation et routing par onglets.
  \item \texttt{lib/services/} : clients HTTP (GitHub, Pixabay). Utilise le package \texttt{http}.
  \item \texttt{lib/models/} : modèles de données pour la désérialisation JSON.
  \item \texttt{lib/widgets/} : composants réutilisables (drawer, animations \texttt{staggered\_fade\_scale}).
\end{itemize}

\section{Dépendances principales}
Extrait depuis \texttt{pubspec.yaml} :
\begin{itemize}[leftmargin=1.2cm]
  \item \texttt{http} : appels REST.
  \item \texttt{video\_player} : lecture vidéo native.
  \item \texttt{animations} : transitions Material motion.
  \item \texttt{curved\_navigation\_bar} : barre de navigation inférieure.
\end{itemize}

\chapter{Configuration des APIs}
\section{Pixabay Video}
\begin{itemize}[leftmargin=1.2cm]
  \item Base : \url{https://pixabay.com/api/videos}
  \item Clé API : stockée dans \texttt{lib/env.dart} (champ \texttt{pexelsApiKey} ; certains libellés UI mentionnent encore \textit{Pexels}, mais l'implémentation est bien sur \textit{Pixabay}).
\end{itemize}

\section{GitHub}
\begin{itemize}[leftmargin=1.2cm]
  \item Endpoint : \url{https://api.github.com/users/mohammedaassou/repos}
  \item Authentification : non requise pour les requêtes publiques de base.
\end{itemize}

\chapter{Interfaces et captures d'écran}
Les captures suivantes illustrent les écrans clefs de l'application.

\begin{figure}[h]
  \centering
  \includegraphics[width=0.92\textwidth]{../demo/1-homepage-minicv.png}
  \caption{Accueil — Mini CV avec animations}
\end{figure}

\begin{figure}[h]
  \centering
  \includegraphics[width=0.92\textwidth]{../demo/2-videos-tab.png}
  \caption{Onglet Recherche Vidéo}
\end{figure}

\begin{figure}[h]
  \centering
  \includegraphics[width=0.92\textwidth]{../demo/3-view-video.png}
  \caption{Lecture vidéo (\texttt{video\_player})}
\end{figure}

\begin{figure}[h]
  \centering
  \includegraphics[width=0.92\textwidth]{../demo/4-repos.png}
  \caption{Liste des dépôts GitHub}
\end{figure}

\begin{figure}[h]
  \centering
  \includegraphics[width=0.92\textwidth]{../demo/5-profile.png}
  \caption{Page Profil (onglets)}
\end{figure}

\chapter{Installation et exécution}
\section{Prérequis}
\begin{itemize}[leftmargin=1.2cm]
  \item Flutter SDK (Dart géré par Flutter).
  \item Un appareil ou émulateur (Android Studio / Xcode / iOS Simulator).
\end{itemize}

\section{Démarrage}
Depuis la racine du projet :
\begin{enumerate}[leftmargin=1.2cm]
  \item Installation des dépendances : \texttt{flutter pub get}
  \item Lancement de l'application : \texttt{flutter run}
\end{enumerate}

\chapter{Tests et qualité}
Des tests de widget de base sont fournis (\texttt{test/widget\_test.dart}). L'application met l'accent sur une architecture claire, la séparation des responsabilités (services, modèles, vues) et l'utilisation de composants réutilisables.

\chapter{Conclusion}
Ce projet illustre une application Flutter moderne intégrant des APIs externes, des animations et des médias. Il constitue une base pédagogique pour approfondir la gestion d'état, l'architecture modulaire, l'optimisation des performances et le déploiement multiplateforme.

\section*{Travaux futurs}
\begin{itemize}[leftmargin=1.2cm]
  \item Internationalisation complète (i18n) et uniformisation des libellés.
  \item Ajout de pagination et de mise en cache pour les listes.
  \item Intégration de thèmes (clair/sombre) et accessibilité.
\end{itemize}

\end{document}
